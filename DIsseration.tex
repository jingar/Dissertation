\documentclass[a4paper,oneside,11pt]{report}
\usepackage[margin=1in]{geometry}
\usepackage[doublespacing]{setspace}
\usepackage[citestyle=authoryear]{biblatex}
\addbibresource{references.bib}
\begin{document}
\tableofcontents

\chapter{Aims and Objectives}
\chapter{Literature Review}
\section{The origin of usability}
Before the 1980s almost all users of computers were highly technical, with much experience and understanding of computing. Beginning in the 1960s a move towards less technical users began with the introduction time sharing and minicomputers. Which was further supplemented In the 1980s due to falling prices of computers as it become possible for many everyday people to become computer users. While falling prices continued, software practices remained the same with implicit assumptions of user experience and knowledge. This lead to frustrated users who lacked the knowledge of computing and become to associate computing with frustrations. From this usability become an important design goal for any system that was to be used by untrained, non technical users. \autocite{gilbert2013}

\section{What is Usability?}
After an examination of various definitions of usability , it reveals a commonality among definitions. 
For instance one of the most used definitions of usability is by Nilesen\autocite{nielsens1993} , defines usability in terms of five attributes Learn-ability, Efficiency. Memorability, Errors and Satisfaction.
\begin{description}
  \item[Learn-ability]How easy a system is to learn for a novice. Generally the first experience a user has of a system is that of learning therefore it is best that this period is kept to a minimum so that the user can be productive as soon as possible.
  \item[Efficiency] How productive a user is once the user has learned how to use the system. There is no point in having a system that once learned provides no benefits to productivity.
  \item[Memorability] How easy it is to remember how to use a system once a user has had some time to use it.  It is important so that an casual user may come back to use the system again and not have to waste time learning the system again.
  \item[Errors] This refers to the error rate of a user when using the system. The error rate should be kept to a minimum and if an error is made the user should be able to recover from them easily. Further care should be taken so that major errors cannot occur.
\end{description}

Similarly the International Organisation of Standardization aka ISO 9241-11\autocite{ISO9241-11} defined usability as \enquote {Extent to which a product can be used by specified users to achieve specified goals with effectiveness, efficiency and satisfaction in a specified context of use.} 
The terms used previous can be further defined:
Effectiveness – The Accuracy and completeness with which users achieve specified goals
Efficiency – The resources expended in relation to the accuracy and completeness with which users achieve specified goals.
Satisfaction – Freedom from discomfort, and positive attitude to the use of the productive
Context of use – Characteristics of the user, tasks and the organizational and physical environments.

ISO further defines the terms goals and users:
Goal – Intended outcome
Task – Activities required to achieve a goal
Furthermore another definition of usability has been offered by Krug, which defines usability as \enquote {making sure that something works well: that a person of average (or even below average) ability and experience can use the 
thing—whether it’s a Web site, a fighter jet, or a revolving door—for its intended purpose without getting hopelessly frustrated.}
 From this major commonalities that can be seen are, that usability is concerned with level of success,satisfaction the user has when interacting with a product.
Former definition of usability by Nielsen is regarded one of the most known definitions but in recent years the ISO definition has been used more in 
While the definitions do not conflict the Nielsen’s definition, the ISO 9241 can be considered to be a better definition on usability as it defines he role of the user as well how the definition of usability depends on context.
\printbibliography
\end{document}

