\documentclass[a4paper,oneside,11pt]{report}
\usepackage[margin=1in]{geometry}
\usepackage[doublespacing]{setspace}
\usepackage[citestyle=authoryear]{biblatex}
\addbibresource{references.bib}
\begin{document}
\tableofcontents

\chapter{Aims and Objectives}
\chapter{Literature Review}
\section{The origin of usability}
Before the 1980s almost all users of computers were highly technical, with much experience and understanding of computing. Beginning in the 1960s a move towards less technical users began with the introduction time sharing and minicomputers. Which was further supplemented In the 1980s due to falling prices of computers as it become possible for many everyday people to become computer users. While falling prices continued, software practices remained the same with implicit assumptions of user experience and knowledge. This lead to frustrated users who lacked the knowledge of computing and become to associate computing with frustrations. From this usability become an important design goal for any system that was to be used by untrained, non technical users.\autocite{gilbert2013}

\section{Usability}
After an examination of various definitions of usability , it reveals a commonality among definitions. 
For instance one of the most used definitions of usability is by Nilesen(\citeyear{nielsens1993}) , defines usability in terms of five attributes Learn-ability, Efficiency. Memorability, Errors and Satisfaction.
\begin{description}
  \item[Learn-ability] How easy a system is to learn for a novice. Generally the first experience a user has of a system is that of learning therefore it is best that this period is kept to a minimum so that the user can be productive as soon as possible.
  \item[Efficiency] How productive a user is once the user has learned how to use the system. There is no point in having a system that once learned provides no benefits to productivity.
  \item[Memorability] How easy it is to remember how to use a system once a user has had some time to use it.  It is important so that an casual user may come back to use the system again and not have to waste time learning the system again.
  \item[Errors] This refers to the error rate of a user when using the system. The error rate should be kept to a minimum and if an error is made the user should be able to recover from them easily. Further care should be taken so that major errors cannot occur.
\end{description}

Similarly the International Organisation of Standardization \autocite{ISO9241-11} defined usability as \enquote {Extent to which a product can be used by specified users to achieve specified goals with effectiveness, efficiency and satisfaction in a specified context of use.} 
The terms used are further defined as follows:

\begin{description}
  \item[Effectiveness] The Accuracy and completeness with which users achieve specified goals
  \item[Efficiency] The resources expended in relation to the accuracy and completeness with which users achieve specified goals.
  \item [Satisfaction] Freedom from discomfort, and positive attitude to the use of the productive
 \item [Context of use] Characteristics of the user, tasks and the organizational and physical environments.
  \item [Goal of use] Intended outcome
  \item [Task of use] Activities required to achieve a goal
\end{description}

Furthermore another, simpler definition of usability has been offered by Krug (\citeyear{krug2005}), which defines usability as \enquote {making sure that something works well: that a person of average (or even below average) ability and experience can use the 
thing—whether it’s a Web site, a fighter jet, or a revolving door—for its intended purpose without getting hopelessly frustrated.}
 From these definitions the major theme that can be drawn out is usability is concerned with level of success,satisfaction the user has when interacting with a product. For this study we will be focusing on the ISO 9241-11 definition out of the three presented. It is the most detailed and widely used among literature. Furthermore the used by the Common Industry Format for Usability Test reports which for further adds to its credibility. From a usability definition one can.
\section{Usability Testing}
Like usability, usability testing as well numerous definitions. Rubin
(\citeyear{rubinchinsel2008}) defines usability testing as \enquote{process that employs people as testing participants who are representative of the target audience to evaluate the degree to which a product meets specific usability criteria.} Another Expert in the field Barnum(2002) defines usability testing as \enquotethe process of learning from users about a product’s usability by observing them using the product.}

\printbibliography
\end{document}